\documentclass[10pt, letterpaper]{article}

% Packages:
\usepackage[
    ignoreheadfoot, % set margins without considering header and footer
    top=2 cm, % seperation between body and page edge from the top
    bottom=2 cm, % seperation between body and page edge from the bottom
    left=2 cm, % seperation between body and page edge from the left
    right=2 cm, % seperation between body and page edge from the right
    footskip=1.0 cm, % seperation between body and footer
    % showframe % for debugging
]{geometry} % for adjusting page geometry
\usepackage{titlesec} % for customizing section titles
\usepackage{tabularx} % for making tables with fixed width columns
\usepackage{array} % tabularx requires this
\usepackage[dvipsnames]{xcolor} % for coloring text
\definecolor{primaryColor}{RGB}{0, 79, 144} % define primary color
\usepackage{enumitem} % for customizing lists
\usepackage{fontawesome5} % for using icons
\usepackage{amsmath} % for math
\usepackage[
    pdftitle={Danila's resume},
    pdfauthor={Danila Danenko},
    pdfcreator={LaTeX with RenderCV},
    colorlinks=true,
    urlcolor=primaryColor
]{hyperref} % for links, metadata and bookmarks
\usepackage[pscoord]{eso-pic} % for floating text on the page
\usepackage{calc} % for calculating lengths
\usepackage{bookmark} % for bookmarks
\usepackage{lastpage} % for getting the total number of pages
\usepackage{changepage} % for one column entries (adjustwidth environment)
\usepackage{paracol} % for two and three column entries
\usepackage{ifthen} % for conditional statements
\usepackage{needspace} % for avoiding page brake right after the section title
\usepackage{iftex} % check if engine is pdflatex, xetex or luatex
\usepackage[T2A]{fontenc}

% Ensure that generate pdf is machine readable/ATS parsable:
\ifPDFTeX
    \input{glyphtounicode}
    \pdfgentounicode=1
    % \usepackage[T1]{fontenc} % this breaks sb2nov
    \usepackage[utf8]{inputenc}
\fi



% Some settings:
\AtBeginEnvironment{adjustwidth}{\partopsep0pt} % remove space before adjustwidth environment
\pagestyle{empty} % no header or footer
\setcounter{secnumdepth}{1} % no section numbering
\setlength{\parindent}{1pt} % no indentation
\setlength{\topskip}{1pt} % no top skip
\setlength{\columnsep}{1cm} % set column seperation
\makeatletter
\let\ps@customFooterStyle\ps@plain % Copy the plain style to customFooterStyle
\patchcmd{\ps@customFooterStyle}{\thepage}{
    \color{gray}\textit{\small Даненко Данила}
}{}{} % replace number by desired string
\makeatother
\pagestyle{customFooterStyle}

\titleformat{\section}{\needspace{4\baselineskip}\bfseries\large}{}{0pt}{}[\vspace{1pt}\titlerule]

\titlespacing{\section}{
    % left space:
    -1pt
}{
    % top space:
    0.5 cm
}{
    % bottom space:
    0.5 cm
} % section title spacing

\renewcommand\labelitemi{$\circ$} % custom bullet points
\newenvironment{highlights}{
    \begin{itemize}[
        topsep=0.20 cm,
        parsep=0.20 cm,
        partopsep=0pt,
        itemsep=0pt,
        leftmargin=0.4 cm + 10pt
    ]
}{
    \end{itemize}
} % new environment for highlights

\newenvironment{highlightsforbulletentries}{
    \begin{itemize}[
        topsep=0.10 cm,
        parsep=0.10 cm,
        partopsep=0pt,
        itemsep=0pt,
        leftmargin=10pt
    ]
}{
    \end{itemize}
} % new environment for highlights for bullet entries


\newenvironment{onecolentry}{
    \begin{adjustwidth}{
        0.2 cm + 0.00001 cm
    }{
        0.2 cm + 0.00001 cm
    }
}{
    \end{adjustwidth}
} % new environment for one column entries

\newenvironment{twocolentry}[2][]{
    \onecolentry
    \def\secondColumn{#2}
    \setcolumnwidth{\fill, 4.5 cm}
    \begin{paracol}{2}
}{
    \switchcolumn \raggedleft \secondColumn
    \end{paracol}
    \endonecolentry
} % new environment for two column entries

\newenvironment{header}{
    \setlength{\topsep}{0pt}\par\kern\topsep\centering\linespread{1.5}
}{
    \par\kern\topsep
} % new environment for the header

\newcommand{\placelastupdatedtext}{% \placetextbox{<horizontal pos>}{<vertical pos>}{<stuff>}
  \AddToShipoutPictureFG*{% Add <stuff> to current page foreground
    \put(
        \LenToUnit{\paperwidth-2 cm-0.2 cm+0.05cm},
        \LenToUnit{\paperheight-1.0 cm}
    ){\vtop{{\null}\makebox[0pt][c]{
        \small\color{gray}\textit{Резюме обновлено 2 апреля 2025}\hspace{\widthof{Резюме обновлено 2 апреля 2025}}
    }}}%
  }%
}%

% save the original href command in a new command:
\let\hrefWithoutArrow\href

% new command for external links:
\renewcommand{\href}[2]{\hrefWithoutArrow{#1}{\ifthenelse{\equal{#2}{}}{ }{#2 }\raisebox{.15ex}{\footnotesize \faExternalLink*}}}


\begin{document}
    \newcommand{\AND}{\unskip
        \cleaders\copy\ANDbox\hskip\wd\ANDbox
        \ignorespaces
    }
    \newsavebox\ANDbox
    \sbox\ANDbox{}

    \placelastupdatedtext
    \begin{header}
        \textbf{\fontsize{24 pt}{24 pt}\selectfont Даненко Данила Андреевич} %заголовок главный

        \vspace{0.3 cm}

        \normalsize
        \mbox{{\color{black}\footnotesize\faMapMarker*}\hspace*{0.13cm}Санкт-Петербург}%
        \kern 0.25 cm%
        \AND%
        \kern 0.25 cm%
        \mbox{\hrefWithoutArrow{mailto: songbirdwhosing@gmail.com}{\color{black}{\footnotesize\faEnvelope[regular]}\hspace*{0.13cm}songbirdwhosing@gmail.com}}%
        \kern 0.25 cm%
        \AND%
        \kern 0.25 cm%
        \mbox{\hrefWithoutArrow{tel:+90-541-999-99-99}{\color{black}{\footnotesize\faPhone*}\hspace*{0.13cm} +79939901409}}%
        \kern 0.25 cm%
        \AND%
        \kern 0.25 cm%
        \mbox{\hrefWithoutArrow{https://github.com/ekkimukk}{\color{black}{\footnotesize\faGithub}\hspace*{0.13cm}ekkimukk}}%
    \end{header}

    \vspace{0.3 cm - 0.3 cm}


    \section{О себе}
    
    \begin{onecolentry}
        \textbf{Я студент Санкт-Петербургского Государственного Университета Телекоммуникаций Им. Проф. М.А. Бонч-Бруевича. Мне нравится программирование, системное администрирование и всё связанное с облачной инфраструктурой. Ищу возможности для практики и дальнейшего профессионального развития, а также хочу применять свои знания для решения реальных задач.}
    \end{onecolentry}

    \vspace{0.3 cm - 0.3 cm}

    \section{Образование}

        \begin{twocolentry}{

        \textit{2022 - 2026}}
            \textbf{Санкт-Петербургский государственный университет телекоммуникаций им. проф. М.А. Бонч-Бруевича}

            \textit{Факультет информационных технологий и программной инженерии (ИТПИ)}

        \vspace{0.10 cm}
        \begin{onecolentry}
            \begin{highlights}

                \item \textbf{Специальность / Профиль:} Программная инженерия / Разработка программного обеспечения и приложений искусственного интеллекта в киберфизических системах
            \end{highlights}
        \end{onecolentry}
        
        \end{twocolentry}

    \vspace{0.3 cm - 0.3 cm}

    \section{Опыт}
    \begin{twocolentry}{
        \textit{Фев 2024 - Наст. вр.}}
        
            \textbf{Linx (ООО «Линкс») | Санкт-Петербург}
                \item Инженер по облачным сервиса
        
        \textit{}

        \begin{onecolentry}
            \begin{highlights}
                \item Проектирование архитектуры, развертывание, внедрение и эксплуатация публичного облака на базе OpenStack и Ceph 
                \item Управление инфраструктурой на базе VMware
                \item Участие в развитии облачных сервисов и миграции локальных сервисов в облако

            \end{highlights}
        \end{onecolentry}
        \end{twocolentry}

    \vspace{0.3 cm - 0.3 cm}

    \section{Дополнительно}
    \begin{twocolentry}{
        \textit{Фев 2024 – Дек 2024}}
        
            \textbf{Курс по основам сетевых технологий}
                \item Завершил курс «Основы сетевых технологий» от Академии Eltex, успешно сдав финальную аттестацию и получив сертификат.

            \textit{}
        

        \begin{onecolentry}
            \begin{highlights}
                \item Знание принципов работы основных сетевых протоколов 
                \item Практический опыт работы с оборудованием Cisco
                \item Понимание модели OSI на всех её уровнях

            \end{highlights}
        \end{onecolentry}
        \end{twocolentry}

    \vspace{0.5 cm}
    
    \begin{twocolentry}{
        \textit{Сен 2024 – Окт 2024}
        \textit{Великий Новгород}}
        
            \textbf{Участник интенсива}
                \item Успешно завершил 26-дневный отборочный интенсив «Школы 21» от Сбера, где разработал 16 индивидуальных и 3 групповых проекта на языке программирования C.

            \textit{}

        \begin{onecolentry}
            \begin{highlights}
                \item Опыт командной разработки
                \item Использование Git и GitLab в рамках реальных проектов
                \item Навык работы с Bash для автоматизации задач
                \item Опыт использования Linux, включая администрирование и разработку

            \end{highlights}
        \end{onecolentry}
        \end{twocolentry}

    \vspace{0.3 cm - 0.3 cm}

    \section{Навыки}
        \begin{onecolentry}
            \textbf{Знание иностранных языков}: Английский - (B1) - Средний
        \end{onecolentry}
         \vspace{0.2 cm}
        
        \begin{onecolentry}
            \textbf{Основные навыки}:
            \colorbox{gray!15}{Java} 
            \colorbox{gray!15}{C/C++} 
            \colorbox{gray!15}{Bash}
            \colorbox{gray!15}{Nix/NixOS}
            \colorbox{gray!15}{Linux}
            \colorbox{gray!15}{MySQL}
            \colorbox{gray!15}{MariaDB}
            \colorbox{gray!15}{Git}
            \colorbox{gray!15}{GitLab}
            \colorbox{gray!15}{GitLab CI/CD}
            \colorbox{gray!15}{Docker}
            \colorbox{gray!15}{Azure}
            \colorbox{gray!15}{Apache}
            \colorbox{gray!15}{Tomcat}
        \end{onecolentry}

\end{document}

